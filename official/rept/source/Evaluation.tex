\chapter{Evaluation}

\section{Overview}
Our system was designed through a combination of computer simulation, and
prototype testing. After carefully designing each subsystem,
per-subsystem PCBs were fabricated and interconnected to build a prototype, which was
thoroughly tested before the final PCB was designed and fabricated.

As our project is a piece of electrical test equipment, the majority of the
tests are electrical in nature. The tests for requirements \hyperref[prs:3.2.2]{3.2.2}
and \hyperref[prs:3.2.3]{3.2.3} (output signal characteristics) require
an oscilloscope with at least $300$~MHz bandwidth. The remaining electrical
tests require only a basic multimeter, and often use the instrument to
verify itself. For example, \hyperref[prs:3.2.4]{3.2.4} (sensitivity) is
verified by measuring the reported amplitude from the output amplifier, and
comparing that to the input noise floor. Requirement \hyperref[prs:3.2.6]{3.2.6}
(accuracy) is verified by examining the normalized sweep of a flat-response
attenuator. Some requirements, for example \hyperref[prs:3.2.1]{3.2.1} (type
of plot), \hyperref[prs:3.3.1]{3.3.1 -- 3.3.3} (interface design), and \hyperref[prs:3.6.4]{3.6.4}
(direct control) are verified simply by observing how the instrument responds to
PC control. Others, for example \hyperref[prs:3.3.4]{3.3.4} (panel connectors),
\hyperref[prs:3.6.1]{3.6.1} (Operator's Manual), \hyperref[prs:3.6.2]{3.6.2} (Protocol Guide),
and \hyperref[prs:3.6.3]{3.6.3} (surface-mount technology) are verified by observing the
instrument and accompanying materials themselves.

The full project requirements can be found in \hyperref[chap:prs]{Appendix~\ref{chap:prs}}, and the full
test procedures can be found in \hyperref[chap:test]{Appendix~\ref{chap:test}}.

\section{Testing and Results}

The full test procedures can be found in Appendix~\ref{chap:test}.

\note{As the full tests have not been performed at the time of this draft,
\emph{Italicized} text is used as a placeholder for missing results.}

\subsection*{Requirement WCP52.3.1.1 --- Required mode: Idle}
This requirement simply means that the device must have a mode in which it is
not performing analysis. \emph{We measured the output signal in this state, and
it measured --- \uV peak to peak. We demonstrated that the ``sample'' annunciator
was not lit.}

\subsection*{Requirement WCP52.3.1.2 --- Required mode: Analysis}
This requirement means that the device must have a non-idle mode in which it is
performing its primary function. This does not require a separate demonstration;
the fact that it performs correctly in the demonstration for WCP52.3.2.6 shows
that it must be capable of performing analysis.

\subsection*{Requirement WCP52.3.2.1 --- Bode plot}
\emph{By providing sample data to the PC software and causing it to display a
plot, we demonstrated that the software is capable of displaying a Bode plot.}

\subsection*{Requirement WCP52.3.2.3 --- Test signal frequency}
We used an oscilloscope to measure the output of the instrument at nominal
frequencies of 1~kHz, 10~MHz, 75~MHz, and 150~MHz. \emph{The maximum deviation from
nominal was ---\%, which is within the 2.5~\% required by the test procedure.}

\subsection*{Requirement WCP52.3.2.3 --- Test signal amplitude}
We used an oscilloscope to measure the amplitude of the instrument's output
at nominal frequencies of 1~kHz, 20~MHz, and 100~MHz. \emph{The measured amplitudes
were, respectively, ---~V~RMS, ---~V~RMS, and ---~V~RMS.} These are all above the
required minimum of 1.25~V~RMS.

\subsection*{Requirement WCP52.3.2.4 --- Sensitivity}
First, we measured the instrument's noise floor at 100~kHz, which is the
absolute limit to its sensitivity. \emph{The noise floor was ---dB.} This
ensures that any measured amplitude above this is the true amplitude, not
the noise floor itself. Then, we measured the relative attenuation of a
40~dB nominal attenuator, verifying that the signal could be seen above the
noise floor. This shows that the instrument is capable of viewing signals
at least 40~dB less than its output amplitude.

\subsection*{Requirement WCP52.3.2.5 --- Extended sensitivity}
The fact that the noise floor above was also under a threshold of -65~dB
satisfies the requirement for extended sensitivity.

\subsection*{Requirement WCP52.3.2.6 --- Accuracy --- Amplitude}
We measured an attenuator specified for \emph{---dB}, and the instrument claimed
an attenuation of \emph{---dB}. This is within the required 3~dB accuracy limit.

\subsection*{Requirement WCP52.3.2.6 --- Accuracy --- Phase}
We measured the phase shift due to propagation delay of a section of RG-316 coaxial
cable. The nominal phase shifts are:

\begin{tabular}{|l|l|}
\hline
Frequency & Phase shift \\ \hline \hline
10~kHz & 0.02\dg \\ \hline
1~MHz & 1.74\dg \\ \hline
10~MHz & 17.4\dg \\ \hline
25~MHz & 43.5\dg \\ \hline
\end{tabular}

We measured phase shifts of:

\begin{tabular}{|l|l|}
\hline
Frequency & Phase shift \\ \hline \hline
10~kHz & 0.02\dg \\ \hline
1~MHz & 1.74\dg \\ \hline
10~MHz & 17.4\dg \\ \hline
25~MHz & 43.5\dg \\ \hline
\end{tabular}

These measurements are within ---\dg of nominal, which satisfies the
5\dg requirement.

\subsection*{Requirement WCP52.3.2.7 --- Extended accuracy}
The fact that the measurements above are also within 1~dB and 1\dg
satisfies the requirement for extended accuracy.

\subsection*{Requirement WCP52.3.2.8 --- Interface safety}
To test this, we issued the \texttt{LOWLEVEL:SET GPIO\_LEVEL} command,
and recieved back the message \texttt{Pin 'GPIO\_LEVEL' is not an output!}. This
verifies that the instrument will not set an output value on an input pin.

\subsection*{Requirement WCP52.3.3.1 --- Interface}
This requirement is satisfied by the fact that WCP52.3.2.1 through WCP52.3.2.8
were satisfied, which all required a PC interface.

\subsection*{Requirement WCP52.3.3.2 --- Communications type}
We connected a serial terminal to the device and issued the \texttt{*IDN?} command,
which produced \emph{the response}. These are both text commands, verifying that the
instrument uses a text-based serial protocol.

\subsection*{Requirement WCP52.3.3.3 --- Communications medium}
The use of a USB-compatible serial terminal in verifying WCP52.3.3.2 also demonstrated
WCP52.3.3.3.

\subsection*{Requirement WCP52.3.3.4 --- Panel connectors}
We demonstrated that the connectors on the front panel are SMA connectors.

\subsection*{Requirement WCP52.3.3.5 --- Auxiliary connector}
We measured the voltages on the power supply pins of the auxiliary connector; they were
\Pos ---~V and \Neg ---~V.

\subsection*{Requirement WCP52.3.6.1 --- Operator's manual}
We presented the PDF operator's manual, and showed the `Theory of Operation' and `Operational
Instructions' sections in it. The latter section had test setups for characterizing both
filters and control loops.

\subsection*{Requirement WCP52.3.6.2 --- Protocol guide}
We showed that there is a protocol guide inside the Operator's Manual, which lists the
SCPI commands and their formats.

\subsection*{Requirement WCP52.3.6.3 --- SMT}
We showed that all parts on the PCB were surface-mount with the following exceptions:

\begin{itemize}
\item[DS1]{--- front panel LED --- exemption: front panel}
\item[J1]{--- power jack --- exemption: connector, front panel}
\item[J2]{--- input jack \#1 --- exemption: connector, front panel}
\item[J3]{--- input jack \#2 --- exemption: connector, front panel}
\item[J4]{--- output jack --- exemption: connector, front panel}
\item[J5]{--- auxiliary jack --- exemption: connector, front panel}
\item[J8]{--- JTAG port --- exemption: connector}
\item[L6]{--- buck-boost inductor --- exemption: inductor $>$ 5~mm diam}
\item[L7]{--- buck inductor --- exemption: inductor $>$ 5~mm diam}
\end{itemize}

Also, we showed that the only leadless parts were those that are directly responsible for the device's function:

\begin{itemize}
\item[U3]{--- DDS --- directly satisfies WCP52.3.2.2}
\end{itemize}

\subsection*{Requirement WCP52.3.6.4 --- Direct control}
We showed the following commands in the protocol guide: \texttt{LOWLEVEL:SET}, \texttt{LOWLEVEL:GET},
\texttt{LOWLEVEL:SPITX}, \texttt{LOWLEVEL:ADC}.

\subsection*{Requirement WCP52.3.6.5 --- Electrical safety}
We measured the voltages present on the pins of the auxiliary connector with a multimeter;
they were, respectively, \emph{3.3~V, 3.3~V, 8.5~V, -9.3~V, 3.3~V, 0.0~V}. We then
measured the maximum amplitude of a signal from the output connector; the maximum peak was
$4.5~V$. These are all within the $15~V$ limit.

\section{Assessment}

In the end, our design met all of the mandatory requirements, and most of the optional ones.
This is overall a useful design. It can measure filters over a wide range with reasonable
accuracy. It is simple to use, robust against input overloads, output overloads, and incorrect
power supply. The system is fully open-source, so users can modify and develop on it as they
please, and students can learn from its operation.

The design has a few drawbacks, however. First, power consumption is high, as the power supply
is inefficient. This was necessary within the development budget and time, as a more efficient
power supply design would also produce more noise that could interfere with the measurements.
Multiple prototypes and significant analysis and testing could have been required. Second,
output phase noise is relatively high at high frequencies. This could cause measurement error
particularly when measuring devices with nonlinearities or when attempting to measure very close
to a strong resonance. However, this noise is inherent in a design with a fixed sample rate, and
correcting this would require a significant increase in complexity of clock generation. Third,
the output can have somewhat significant DC offset voltages (up to around 100mV), which is not
a huge problem but could be surprising to some users. A coupling capacitor after the differential
amplifier stage would fix this, but using the correct one would require significant testing and
would risk introducing amplitude loss at either the high or the low end. Fourth, the frontend
is not selective; it always measures the total input power over its full bandwidth. A frontend
with a selective mixer to measure only signals at the same frequency as the output frequency
would greatly increase the measurement noise floor and improve the behavior with nonlinear
devices-under-test.

