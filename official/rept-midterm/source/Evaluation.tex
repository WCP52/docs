\chapter{Evaluation}

\section{Overview}
Our system was designed through a combination of computer simulation, and
prototype testing. After carefully designing each subsystem, we fabricated
per-subsystem PCBs and interconnected them to build a prototype, which was
thoroughly tested before the final PCB was designed and fabricated.

As our project is a piece of electrical test equipment, the majority of the
tests will be electrical in nature. The tests for requirements \hyperref[prs:3.2.2]{3.2.2}
and \hyperref[prs:3.2.3]{3.2.3} (output signal characteristics) will require
an oscilloscope with at least $300$~MHz bandwidth. The remaining electrical
tests require only a basic multimeter, and will often use the instrument to
verify itself. For example, \hyperref[prs:3.2.4]{3.2.4} (sensitivity) is
verified by measuring the reported amplitude from the output amplifier, and
comparing that to the input noise floor. Requirement \hyperref[prs:3.2.6]{3.2.6}
(accuracy) is verified by examining the normalized sweep of a flat-response
attenuator. Some requirements, for example \hyperref[prs:3.2.1]{3.2.1} (type
of plot), \hyperref[prs:3.3.1]{3.3.1 -- 3.3.3} (interface design), and \hyperref[prs:3.6.4]{3.6.4}
(direct control) can be verified simply by observing how the instrument responds to
PC control. Others, for example \hyperref[prs:3.3.4]{3.3.4} (panel connectors),
\hyperref[prs:3.6.1]{3.6.1} (Operator's Manual), \hyperref[prs:3.6.2]{3.6.2} (Protocol Guide),
and \hyperref[prs:3.6.3]{3.6.3} (surface-mount technology) can be verified by observing the
instrument and accompanying materials themselves.

The full project requirements can be found in \hyperref[chap:prs]{Appendix~\ref{chap:prs}}, and the full
test procedures can be found in \hyperref[chap:test]{Appendix~\ref{chap:test}}.

\section{Prototype}
\greylipsum{1-6}

\section{Testing and Results}
\greylipsum{7}

\subsection{Requirement WCP34.1 --- Weight}
\greylipsum{8}

\section{Assessment}
\greylipsum{1-5}
