\chapter{Test Procedures}
\label{chap:test}

\section*{3.1.1: Required mode: Idle}
The idle state of the signal source is to be verified by viewing the signal from the
``output'' connector on an oscilloscope. Any signal present must be less than $5$~V peak to peak.

The idle state of the detection subsystem is verified by viewing the status annunciators on
the printed circuit board. The annunciator designated ``sample'' must not light.


\section*{3.1.2: Required mode: Analysis}
This requirement is satisfied peripherally by the completion of \hyperref[tp:3.2.1]{requirement 3.2.1}.

\section*{3.2.1: Bode plot}
\label{tp:3.2.1}
To test this requirement, connect a patch cable between the ``Output'' connector and the
``Input 1'' connector, and initiate a non-normalized single-input analysis via the PC
software with frequency bounds of $1$~kHz and $150$~MHz. A pair of plots must
be produced, one indicating gain, and one indicating phase. This requirement places no
constraints on the data that is displayed.

\section*{3.2.2: Test signal frequency}
\label{tp:3.2.2}
To test this requirement, connect a patch cable between the ``Output'' connector and an oscilloscope
with at least $100$~MHz bandwidth. Either using the PC software or manual control via a
serial terminal, command the instrument to generate a signal with a frequency of $1$~kHz, and then
a signal with a frequency of $150$~MHz. Using the oscilloscope's frequency display, verify
that the signal is present, and that the frequency is within $2.5$~\% of its nominal value.

\section*{3.2.3: Test signal amplitude}
To test this requirement, connect a patch cable between the ``Output'' connector and an
oscilloscope with at least $300$~MHz bandwidth. Either using the PC software or manual control
via a serial terminal, command the instrument to generate a signal with a frequency of $1$~kHz,
and then a signal with a frequency of $100$~MHz, both with maximum amplitude. Using the
oscilloscope's amplitude display, verify that both signals have an amplitude of at least
$1.25$~V~RMS.

\section*{3.2.4: Sensitivity}
\label{tp:3.2.4}
To test this requirement, connect a patch cable between the ``Output'' connector and
the ``Input 1'' connector. Command the instrument to generate a signal with a frequency of
$100\;\mr kHz$ and maximum amplitude. Query the reported amplitude on the input. Next,
remove the patch cable, and query the reported amplitude on the input again. The reported
amplitude must be at least $45\;\mr dB$ lower than the first reported amplitude.

\section*{3.2.5: Extended sensitivity}
To test this optional requirement, perform the same test as in \hyperref[tp:3.2.4]{requirement 3.2.4},
but verify that the final reported amplitude is at least $65\;\mr dB$ lower than the first reported amplitude.

\section*{3.2.6: Accuracy}
\label{tp:3.2.6}
To test this requirement, connect a pair of patch cables in series using a cable coupler,
and connect this pair between the ``Output'' connector and the ``Input 1'' connector. Configure
the PC software for a full sweep from $1\;\mr kHz$ to $150\;\mr MHz$ including phase analysis.
Perform normalization. Then, remove the cable coupler and insert a $50\;\mr\Omega$ attenuator with
specified attenuation between $5\;\mr dB$ and $15\;\mr dB$ and bandwidth of at least $200\;\mr MHz$.
Perform analysis. Verify that the reported gain is within $3\;\mr dB$ of the specified gain of the
attenuator at all frequencies, and that the phase shift is within $5\dg$ of zero
(above $-5\dg$ and below $+5\dg$) at all frequencies.

\section*{3.2.7: Extended accuracy}
To test this optional requirement, perform the test for \hyperref[tp:3.2.6]{requirement 3.2.6},
with the following modifications: the upper sweep limit should be $20$~MHz, reported gain must be
within $1$~dB of the specified attenuator gain, and phase shift must be within $1\dg$ of zero
(above $-1\dg$ and below $+1\dg$) at all frequencies.

\section*{3.2.8: Interface safety}
\label{tp:3.2.8}
To test this requirement, use the interface control commands to attempt to set a GPIO pin
which serves as a signal input to be an output instead. The system must respond with an error.

\section*{3.3.1: Interface}
This requirement is satisfied peripherally by the completion of requirements \hyperref[tp:3.2.1]{3.2.1}
through \hyperref[tp:3.2.8]{3.2.8}, which all required PC interfacing to complete.

\section*{3.3.2: Communications type}
To test this requirement, connect the instrument to a PC. Connect a serial terminal application
to it, and perform at least one command that produces a response. Demonstrate that the
command and response comprise displayable text symbols.

\section*{3.3.3: Communications medium}
To test this requirement, first demonstrate that the PC connector is a standard
USB connector. Then, connect the instrument to a PC running a Linux-based
operating system, and follow this procedure:

\begin{enumerate}
\item{Open a command prompt.}
\item{Run this command to discover the device ID: \texttt{dmesg | tail}}
\item{Enter the following directory: \texttt{/sys/bus/usb/devices/}\emph{deviceID}\texttt{/driver/module/drivers}}
\item{View the directory listing, and verity that at least one of the following entries is present:
    \begin{itemize}
    \item{\texttt{usb-serial:generic}}
    \item{\texttt{usb:usbserial}}
    \item{\texttt{usb:usbserial\_generic}}
    \end{itemize}
}
\end{enumerate}

\section*{3.3.4: Panel connectors}
To test this requirement, observe the front panel of the device, and see that the RF connectors are either
SMA or BNC.

\section*{3.3.5: Auxiliary connector}
To test this requirement, ensure that the device is powered and connected to a PC. Connect the
``common'' input of a multimeter to device ground, then use the multimeter to probe the ``Auxiliary'' connector
and verify that there are power supply pins with a voltage present of at least $+7\;\mr V$ and
$-7\;\mr V$. Then, switch the multimeter to ammeter mode, and connect in series a selected resistor. Probe
these pins again, and verify that at least $40\;\mr mA$ is supplied.

The resistor is to be selected such that it will draw at least $40\;\mr mA$ at the voltage that was
measured on the connectors. This resistance may depend on the actual voltage that was present,
which is allowed to be \emph{above} the required $\pm 7\;\mr V$.

\section*{3.6.1: Operator's Manual}
Demonstrate the existence of a manual in digital format, containing at least the following
sections: Theory of Operation, Operational Instructions. Demonstrate that there exist example
test setups for characterization of filters and of control loops.

\section*{3.6.2: Protocol Guide}
Demonstrate the existence of documentation explaining the protocol with which a PC communicates with the
instrument. This documentation may be part of the Operator's Manual, or it may be separate.

\section*{3.6.3: SMT}
Demonstrate that the parts on the PCB are surface-mount devices, with the following allowed
exceptions:
\begin{itemize}
\item{Connectors: through-hole connections have higher mechanical stability}
\item{Inductors and capacitors larger than $5\;\mr{mm}$ in diameter: through-hole parts have higher mechanical stability}
\end{itemize}

Demonstrate that the only leadless devices on the PCB are essential for its function. This may be
a result of directly satisfying an above requirement. An anticipated example is device \refdes{U3},
an Analog Devices AD9958BCPZ digital frequency synthesizer, which directly satisfies
\hyperref[tp:3.2.2]{requirement 3.2.2} (test signal frequency between $1\;\mr{kHz}$ and $150\;\mr MHz$).

\section*{3.6.4: Direct control}
Viewing the protocol guide, demonstrate the existence of:

\begin{itemize}
\item{A command to set the output value of an arbitrary GPIO pin.}
\item{A command to query the input value of an arbitrary GPIO pin.}
\item{A command to transmit arbitrary data via the on-board SPI interface.}
\item{A command to query the direct output value of the analog-to-digital converter.}
\end{itemize}

\section*{3.6.5: Electrical safety}
To test this requirement, connect the ``common'' input of a multimeter to the instrument ground, and
verify that no signals on the ``Auxiliary'' connector are more positive than $15$~V or more negative than $-15$~V.
Then, connect a patch cable between the ``Output'' connector and an oscilloscope. Do not terminate the end. Command
the instrument to generate a frequency of $100$~kHz and maximum amplitude. Using the oscilloscope's amplitude display,
verify that the positive and negative peaks are no more positive than $15$~V and no more negative than
$-15$~V.
