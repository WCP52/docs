\chapter{Project Requirements}
\label{chap:prs}

\section*{3.1: Required States and Modes}

This system requires the following modes:

\subsection*{3.1.1: Idle}
\label{prs:3.1.1}
The signal source and detection subsystems are inactive, and the system is waiting for commands.

\subsection*{3.1.2: Analysis}
\label{prs:3.1.2}
The system is performing a gain/phase analysis.

\section*{3.2: System Capability Requirements}

\subsection*{3.2.1: Bode plot}
\label{prs:3.2.1}
The analyzer shall be able to display a plot on the operator's PC in the form of a Bode plot.

\subsection*{3.2.2: Test signal frequency}
\label{prs:3.2.2}
The analyzer shall be capable of sourcing test signals between $1$~kHz and $150$~MHz.
(Not applicable: state \hyperref[prs:3.1.1]{WCP52-3.1.1 --- idle})

\subsection*{3.2.3: Test signal amplitude}
\label{prs:3.2.3}
The analyzer shall be capable of output amplitudes up to $1.25$~V~RMS at frequencies up to
$100$~MHz. (Not applicable: state \hyperref[prs:3.1.1]{WCP52-3.1.1 --- idle})

\subsection*{3.2.4: Sensitivity}
\label{prs:3.2.4}
The analyzer shall be able to detect signals down to at least $40$~dB below the output
amplitude. (Not applicable: state \hyperref[prs:3.1.1]{WCP52-3.1.1 --- idle})

\subsection*{3.2.5: Extended sensitivity}
\label{prs:3.2.5}
The analyzer should be able to detect signals down to at least $60$~dB below the output
amplitude. (Not applicable: state \hyperref[prs:3.1.1]{WCP52-3.1.1 --- idle})

\subsection*{3.2.6: Accuracy}
\label{prs:3.2.6}
Amplitude accuracy shall be within $3$~dB, and phase accuracy within $5\dg$.

\subsection*{3.2.7: Extended accuracy}
\label{prs:3.2.7}
Amplitude accuracy should be within $1$~dB, and phase accuracy within $1\dg$, for
frequencies less than $20$~MHz.

\subsection*{3.2.8: Interface safety}
\label{prs:3.2.8}
The hardware shall not be able to be damaged by its remote interface, unless an ``unlock'' command
has been issued.

\section*{3.3: System External Interface Requirements}

\subsection*{3.3.1: Interface}
\label{prs:3.3.1}
The analyzer shall interface with a PC.

\subsection*{3.3.2: Communications type}
\label{prs:3.3.2}
The analyzer should use a text-driven protocol.

\subsection*{3.3.3: Communications medium}
\label{prs:3.3.3}
The analyzer should use a common, standard communication protocol, for example, USB-CDC.

\subsection*{3.3.4: Panel connectors}
\label{prs:3.3.4}
The analyzer shall use either SMA or BNC connectors to interface to the device under test (DUT).

\subsection*{3.3.5: Auxiliary connector}
\label{prs:3.3.5}
The analyzer shall provide power via a front-panel connection for use with external DUT
adapters. The voltage should be at least $\pm 7$~V, and up to $40$~mA should be available.

\section*{3.4: System Internal Interface Requirements}
All internal interfaces are left to the system designers.

\section*{3.5: System Internal Data Requirements}
All decisions about internal data are left to the system designers.

\section*{3.6: Other System Requirements}

\subsection*{3.6.1: Operator's Manual}
\label{prs:3.6.1}
The system should include a simple operator's manual, which should include a brief
Theory of Operation explaining its design, instructions for using each function,
and example test setups for characterization of filters and control loops.

\subsection*{3.6.2: Protocol Guide}
\label{prs:3.6.2}
The system shall include a protocol guide, showing how to communicate with it.

\subsection*{3.6.3: SMT}
\label{prs:3.6.3}
The PCB shall be produced using surface-mount technology as much as is reasonable,
without no-lead packages unless absolutely required.

\subsection*{3.6.4: Direct control}
\label{prs:3.6.4}
The interface should expose direct control of the hardware functions, allowing additional
features to be implemented.

\subsection*{3.6.5: Electrical safety}
\label{prs:3.6.5}
The system shall have no voltages greater than $30$~V peak-to-peak accessible
externally.

\section*{3.7: Precedence and Criticality of Requirements}

All requirements have equal weight.
